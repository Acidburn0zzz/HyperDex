\chapter{Installing HyperDex}

HyperDex provides two means of installation for users.  The easiest and most
convenient way to install HyperDex is to use precompiled binary packages.
Packages are available for a several platforms and more platforms will be
supported as resources permit.  For users who need more control over their
installation, HyperDex provides the option to compile from source tarballs.

\section{Installing Binary Packages}

The HyperDex team maintains repositories for Debian, Ubuntu, and Fedora so that
the latest version is always conveniently available.

\subsection{Debian}

To access the Debian repository, add the following to
\texttt{/etc/apt/sources.list}:

\begin{consolecode}
deb http://debian.hyperdex.org wheezy main
\end{consolecode}

Subsequent invocations of the package manager may complain about the absence of
the relevant package signing key.  You can download the
\href{http://debian.hyperdex.org/hyperdex.gpg.key}{Debian public key} and add
it with:

\begin{consolecode}
apt-key add hyperdex.gpg.key
\end{consolecode}

The following code may be copied and pasted into a root terminal to quickly
setup the Debian repository and install HyperDex:

\begin{consolecode}
wget -O - http://debian.hyperdex.org/hyperdex.gpg.key \
| apt-key add -
wget -O /etc/apt/sources.list.d/hyperdex.list \
    http://debian.hyperdex.org/hyperdex.list
aptitude update
aptitude install hyperdex
\end{consolecode}

\subsection{Ubuntu}

To access the Ubuntu repository, add the following to
\texttt{/etc/apt/sources.list}:

\begin{consolecode}
deb [arch=amd64] http://ubuntu.hyperdex.org precise main
\end{consolecode}

Subsequent invocations of the package manager may complain about the absence of
the relevant package signing key.  You can download the
\href{http://ubuntu.hyperdex.org/hyperdex.gpg.key}{Ubuntu public key} and add
it with:

\begin{consolecode}
apt-key add hyperdex.gpg.key
\end{consolecode}

The following code may be copied and pasted to quickly setup the Ubuntu
repository and install HyperDex:

\begin{consolecode}
sudo wget -O /etc/apt/sources.list.d/hyperdex.list \
    http://ubuntu.hyperdex.org/hyperdex.list
sudo wget -O - http://ubuntu.hyperdex.org/hyperdex.gpg.key \
| sudo apt-key add -
sudo apt-get update
sudo apt-get install hyperdex
\end{consolecode}

\subsection{Fedora}

To access the Fedora repository, add the following to \texttt{/etc/yum.conf}:

\begin{consolecode}
[hyperdex]
name=hyperdex
baseurl=http://fedora.hyperdex.org/
enabled=1
gpgcheck=0
\end{consolecode}

To install HyperDex, issue the following command as root:

\begin{consolecode}
yum install hyperdex
\end{consolecode}

\subsection{Package Names}

Packages are named consistently across platforms.  On all architectures, the
\texttt{hyperdex} meta-package pulls in all HyperDex packages.  For most users,
installing the \texttt{hyperdex} package and any language packages will be
sufficient.

For completeness, here is a list of all packages:

\begin{description}
\item[\texttt{hyperdex-daemon}]
This package contains the HyperDex daemon that runs on each storage node.  It
required on every storage node.

\item[\texttt{libhyperdex-client}]
This package contains the client library for C/C++ bindings.  It is required on
all platforms which will access HyperDex.

On Debian and Ubuntu systems, this will have a small number appended to the
package name indicating the version of the package contained within.

\item[\texttt{python-hyperdex-client}]
This provides the python module \texttt{hyperdex.client}.  This package is only
required for systems that need to interact with HyperDex from Python.

\item[\texttt{libhyperdex-coordinator}]
This package provides the coordinator for a HyperDex cluster.  This package is
required only on systems which will serve as the coordinator for the cluster.

\item[\texttt{replicant}]
This package provides part of the HyperDex coordinator and is only necessary on
systems which will serve as the coordinator for the cluster.
\end{description}

Most packages are complemented by development and debug packages.  In the
development package, there are header files and static libraries.  The debug
packages provide symbols which will aid in providing tracebacks to the HyperDex
developers.  Please consult your package manager to find these packages.

\section{Installing Source Tarballs}

An alternative to installing a prepackaged binary is to install from source
tarballs.  This is a straightforward process that should work on most any recent
Linux distribution for which there isn't a prepackaged binary.  We'll first list
the prerequisites to installing HyperDex from a source tarball.  Then, we'll
describe how to configure HyperDex.  Finally, we'll describe the installation
step.

\subsection{Prerequisites}

HyperDex has a minimal number of prerequisites for installation from source.
Although we list all prerequisites in this section for completeness, the
HyperDex configuration step will automatically warn about any missing
dependencies.

Required Dependencies:

\begin{itemize}
\item \href{http://code.google.com/p/cityhash/}{Google CityHash}:  Used for
    hashing strings.  Requires version 1.0.x
\item \href{http://code.google.com/p/google-glog/}{Google Glog}:  Used for
    logging.  Requires version 0.3.x.
\item \href{https://code.google.com/p/sparsehash/}{Google SparseHash}:  Used for
    hash tables.  Requires version 2.0.x.
\item \href{http://rpm5.org/}{libpopt}: Used for argument parsing.  The
    developers use 1.16 but any recent version should do.
\item \href{http://hyperdex.org/downloads/}{libpo6}: Used for general POSIX
    wrappers.  Requires the latest version.  This package is maintained by the
    HyperDex developers.
\item \href{http://hyperdex.org/downloads/}{libe}: Used for general C++:
    utilities.  Requires the latest version.  This package is maintained by the
    HyperDex developers.
\item \href{http://hyperdex.org/downloads/}{BusyBee}: Used for server-server
    and client-server communication.  Requires the latest version.  This package
    is maintained by the HyperDex developers.
\item \href{http://hyperdex.org/downloads/}{HyperLevelDB}  Used to store data
    on the servers.
\item \href{http://hyperdex.org/downloads/}{Replicant} Used for making the
    coordinator fault tolerant.  Requires the latest version.  This package is
    maintained by the HyperDex developers.
\end{itemize}

Dependencies for Python Bindings:

\begin{itemize}
\item \href{http://python.org/}{Python}: Version 2.6 or 2.7 with the
    development headers installed.
\end{itemize}

Dependencies for Java Bindings:

\begin{itemize}
\item \href{http://openjdk.java.net/}{Java}: We test against OpenJDK 6.  Your
    system must include \texttt{javac}, \texttt{jar}, and the JNI development
    headers.
\item \href{http://www.swig.org/}{SWIG}: Used to generate part of the
    bindings.  We test SWIG 2.0.
\end{itemize}

Dependencies for Yahoo! Cloud Serving Benchmark (YCSB):

\begin{itemize}
\item \href{https://github.com/brianfrankcooper/YCSB/wiki}{YCSB} The YCSB
    distribution is a moving target.  We generally build against the latest Git
    release.
\end{itemize}

\subsection{Configuring}

HyperDex uses the Autotools to make configuration and installation as
straightforward as possible.  After extracting the HyperDex tarball, you'll need
to configure HyperDex.  The simplest configuration installs HyperDex in its
default location (\texttt{/usr/local}) using the C++ compiler found on the
system.  The configuration is performed in the directory extracted from the
tarball and looks like:

\begin{consolecode}
./configure
\end{consolecode}

This basic configuration will configure the HyperDex daemon and native client
library components to be built; however it omits several useful options for
configuring HyperDex.  The rest of this section will highlight common ways to
configure HyperDex.  Unless otherwise noted, all options should work well
together.

\subsubsection{Java Bindings}

HyperDex does not build Java bindings by default.  To enable the Java bindings,
you must pass \texttt{--enable-java-bindings} to \texttt{./configure} like so:

\begin{consolecode}
./configure --enable-java-bindings
\end{consolecode}

If any of the prerequisites are missing \texttt{./configure} should fail.

\begin{quote}
Java bindings are currently unavailable in \HyperDexVersion.
\end{quote}

\subsubsection{Python Bindings}

HyperDex does not build Python bindings by default.  To enable the Python
bindings, you must pass \texttt{--enable-python-bindings} to
\texttt{./configure} like so:

\begin{consolecode}
./configure --enable-python-bindings
\end{consolecode}

If Python or its headers cannot be found, \texttt{./configure} will fail.

\subsubsection{Yahoo! Cloud Serving Benchmark}

HyperDex provides all the source code necessary to build a HyperDex driver
for the YCSB benchmark.  If Java bindings are enabled, then YCSB can be built
with \texttt{--enable-ycsb-driver}.

\begin{consolecode}
./configure --enable-ycsb-driver
\end{consolecode}

Note that YCSB must be in your Java \texttt{CLASSPATH}.  Configure cannot detect
YCSB by itself.

\subsubsection{Changing the Installation Directory}

By default HyperDex installs files in \texttt{/usr/local}.  If you'd like to
install HyperDex elsewhere, you can specify the installation prefix at configure
time.  For example, to install HyperDex in \texttt{/opt/hyperdex}:

\begin{consolecode}
./configure --prefix=/opt/hyperdex
\end{consolecode}

Check the \texttt{--help} option to configure for more ways to tune where
HyperDex places files.

\subsection{Installing}

Once configured, HyperDex is simple to build and install.  Keep in mind that the
following commands may fail if the installation directory is not writable by the
current user.

\begin{consolecode}
make
make install
ldconfig
\end{consolecode}

\section{Installing from Git}

Developers wishing to contribute to the development of HyperDex may build
HyperDex directly from Git.  Building from Git is as straightforward as building
from source tarballs, but requires a few extra dependencies and some setup
before the \texttt{./configure} step.

In order to build the HyperDex repository, you'll need to have the following
utilities installed.  Most of these utilities are prepacked for Linux
distributions.  Note that since these dependencies are only required for
building from Git, they will not be detected at \texttt{./configure} time and
instead \texttt{make} will fail with an error message.

Required Dependencies:

\begin{itemize}
\item \href{http://www.gnu.org/software/autoconf/}{Autoconf} Used as part of
    the build system.  Required for all builds.
\item \href{http://www.gnu.org/software/automake/}{Automake} Used as part of
    the build system.  Required for all builds.
\item \href{http://www.gnu.org/software/libtool/}{Libtool} Used as part of the
    build system.  Required for all builds.
\item \href{http://www.gnu.org/software/autoconf-archive/}{Autoconf Archive}
    Used as part of the build system.  Required for all builds.
\item \href{http://www.freedesktop.org/wiki/Software/pkg-config/}{pkg-config}
    Used as part of the build system.  Required for all builds.
\item \href{http://flex.sourceforge.net/}{Flex} Used for building internal
    parsers.  Required for all builds.
\item \href{http://www.gnu.org/software/bison/}{Bison} Used for building
    internal parsers.  Required for all builds.
\item \href{http://cython.org/}{Cython} Used for building Python bindings.
    Required for \texttt{--enable-python-bindings}.
    Recommmended version: $\ge$ 0.15.
\item \href{http://www.gnu.org/software/gperf/}{Gperf}  Generate perfect
    hashes.  Used in the client library.
\end{itemize}

You'll need to build po6, e, BusyBee, HyperLevelDB Replicant, and HyperDex from
Git, as the development version often introduces across repositories.

For each of these repositories, you may build and install the code with:

\begin{consolecode}
autoreconf -i
./configure
make
make install
ldconfig
\end{consolecode}

\section{Verifying Installation}

Once you have HyperDex installed, you should be able to view the built-in help.
If the following commands provide meaningful output, then it is very likely that
HyperDex is installed correctly and ready for use.

\begin{consolecode}
hyperdex help
hyperdex daemon --help
\end{consolecode}
