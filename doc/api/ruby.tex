\chapter{Ruby API}

\section{Client Library}

HyperDex provides ruby bindings under the module \code{HyperDex::Client}.  This
library wraps the HyperDex C Client library and enables use of native Ruby data
types.

This library was brought up-to-date following the 1.0.rc5 release.

\subsection{Building the HyperDex Ruby Binding}

The HyperDex Ruby Binding must be requested at configure time as it is not
automatically built.  You can ensure that the Ruby bindings are always built by
providing the \code{--enable-ruby-bindings} option to \code{./configure} like
so:

\begin{consolecode}
% ./configure --enable-client --enable-ruby-bindings
\end{consolecode}

\subsection{Using Ruby Within Your Application}

All client operation are defined in the \code{HyperDex::Ruby} module.  You can
access this in your program with:

\begin{rubycode}
require 'hyperdex'
\end{rubycode}

\subsection{Hello World}

The following is a minimal application that stores the value "Hello World" and
then immediately retrieves the value:

\inputminted{ruby}{api/ruby/hello-world.rb}

You can run this example with:

\begin{consolecode}
% ruby hello-world.rb
put "Hello World!"
got:
{:v=>"Hello World!"}
\end{consolecode}

Right away, there are several points worth noting in this example:

\begin{itemize}
\item Every operation is synchronous.  The PUT and GET operations run to
completion by default.

\item Ruby types are automatically converted to HyperDex types.  There's no need
to specify information such as the length of each string, as one would do with
the C API.

\item Ruby symbols are permitted wherever a string may be used.  By convention,
space names and attribute names are specified using Ruby symbols, e.g.
\code{:kv} and \code{:v}.  Of course, you may use strings for these parameters
too.
\end{itemize}

\subsection{Asynchronous Operations}

For convenience, the Ruby bindings treat every operation as synchronous.  This
enables you to write short scripts without concern for asynchronous operations.
Most operations come with an asynchronous form, denoted by the \code{async\_}
prefix.  For example, the above Hello World example could be rewritten in
asynchronous fashion as such:

\inputminted{ruby}{api/ruby/hello-world-async-wait.rb}

This enables applications to issue multiple requests simultaneously and wait for
their completion in an application-specific order.  It's also possible to use
the \code{loop} method on the client object to wait for the next request to
complete:

\inputminted{ruby}{api/ruby/hello-world-async-loop.rb}

\subsection{Data Structures}

The Ruby bindings automatically manage conversion of data types from Ruby to
HyperDex types, enabling applications to be written in idiomatic Ruby.

\subsubsection{Examples}

This section shows examples of Ruby data structures that are recognized by
HyperDex.  The examples here are for illustration purposes and are not
exhaustive.

\paragraph{Strings}

The HyperDex client recognizes Ruby's strings and symbols and automatically
converts them to HyperDex strings.  For example, the following two calls are
equivalent and have the same effect:

\begin{rubycode}
c.put("kv", "somekey", {"v" => "somevalue"})
c.put(:kv, :somekey, {:v => :somevalue})
\end{rubycode}

The recommended convention is to use symbols for space and attribute names, and
strings for keys and values like so:

\begin{rubycode}
c.put(:kv, "somekey", {:v => "somevalue"})
\end{rubycode}

\paragraph{Integers}

The HyperDex client recognizes Ruby's integers, longs, and fixnums and
automatically converts them to HyperDex integers.  For example:

\begin{rubycode}
c.put(:kv, "somekey", {:v => 42})
\end{rubycode}

\paragraph{Floats}

The HyperDex client recognizes Ruby's floating point numbers and automatically
converts them to HyperDex floats.  For example:

\begin{rubycode}
c.put(:kv, "somekey", {:v => 3.1415})
\end{rubycode}

\paragraph{Lists}

The HyperDex client recognizes Ruby lists and automatically converts them to
HyperDex lists.  For example:

\begin{rubycode}
c.put(:kv, "somekey", {:v1 => ["a", "b", "c"]})
c.put(:kv, "somekey", {:v2 => [1, 2, 3]})
c.put(:kv, "somekey", {:v3 => [1.0, 0.5, 0.25]})
\end{rubycode}

\paragraph{Sets}

The HyperDex client recognizes Ruby sets and automaticaly converts them to
HyperDex sets.  For example:

\begin{rubycode}
require 'set'
c.put(:kv, "somekey", {:v1 => (Set.new ["a", "b", "c"])})
c.put(:kv, "somekey", {:v2 => (Set.new [1, 2, 3])})
c.put(:kv, "somekey", {:v3 => (Set.new [1.0, 0.5, 0.25])})
\end{rubycode}

Note that you'll have to include the set module from the standard library.

\paragraph{Maps}

The HyperDex client recognizes Ruby hashes and automatically converts them to
HyperDex maps.  For example:

\begin{rubycode}
require 'set'
c.put(:kv, "somekey", {:v1 => {"k" => "v"}})
c.put(:kv, "somekey", {:v2 => {1 => 2}})
c.put(:kv, "somekey", {:v3 => {3.14 => 0.125}})
c.put(:kv, "somekey", {:v3 => {"a" => 1}})
\end{rubycode}

\subsection{Attributes}

Attributes in Ruby are specified in the form of a hash from attribute names to
their values.  As you can see in the examples above, attributes are specified in
the form:

\begin{rubycode}
{:name => "value"}
\end{rubycode}

\subsection{Map Attributes}

Map attributes in Ruby are specified in the form of a nested hash.  The outer
hash key specifies the name, while the inner hash key-value pair specifies the
key-value pair of the map.  For example:

\begin{rubycode}
{:name => {"key" => "value"}}
\end{rubycode}

\subsection{Predicates}

Predicates in Ruby are specified in the form of a hash from attribute names to
their predicates.  In the simple case, the predicate is just a value to be
compared against:

\begin{rubycode}
{:v => "value"}
\end{rubycode}

This is the same as saying:

\begin{rubycode}
{:v => HyperDex::Client::Equals.new('value')}
\end{rubycode}

The Ruby bindings support the full range of predicates supported by HyperDex
itself.  For example:

\begin{rubycode}
{:v => HyperDex::Client::LessEqual.new(5)}
{:v => HyperDex::Client::GreaterEqual.new(5)}
{:v => HyperDex::Client::Range.new(5, 10)}
{:v => HyperDex::Client::Regex.new('^s.*')}
{:v => HyperDex::Client::LengthEquals.new(5)}
{:v => HyperDex::Client::LengthLessEqual.new(5)}
{:v => HyperDex::Client::LengthGreaterEqual.new(5)}
{:v => HyperDex::Client::Contains.new('value')}
\end{rubycode}

\subsection{Error Handling}

All error handling within the Ruby bindings is done via the
\code{begin}/\code{rescue} mechanism of Ruby.  Errors will be raised by the
library and should be handled by your application.  For example, if we were
trying to store an integer (5) as attribute \code{:v}, where \code{:v} is
actually a string, we'd generate an error.

\begin{rubycode}
begin
    puts c.put(:kv, :my_key, {:v => 5})
rescue HyperDex::Client::HyperDexClientException => e
    puts e.status
    puts e.symbol
    puts e
end
\end{rubycode}

Errors of type \code{HyperDexClientException} will contain both a message
indicating what went wrong, as well as the underlying \code{enum
hyperdex\_client\_returncode}.  The member \code{status} indicates the numeric
value of this enum, while \code{symbol} returns the enum as a string.  The above
code will fail with the following output:

\begin{verbatim}
8525
HYPERDEX_CLIENT_WRONGTYPE
invalid attribute "v": attribute has the wrong type
\end{verbatim}

\subsection{Operations}

% This LaTeX file is generated by bindings/ruby/gen_doc.py

\paragraph{\code{get}}
\index{get!Ruby API}
\begin{ccode}
Client :: get(spacename, key)
\end{ccode}
\funcdesc Retreive the object with key "key" from space "space".

\noindent\textbf{Cost:}  Approximately one network round trip.


\noindent\textbf{Consistency:}  Linearizable



\noindent\textbf{Parameters:}
\begin{description}[labelindent=\widthof{{\code{spacename}}},leftmargin=*,noitemsep,nolistsep,align=right]
\item[\code{spacename}] The name of the space as a string or symbol.
\item[\code{key}] The key for the operation as a Ruby type
\end{description}

\noindent\textbf{Returns:}
Object if found, nil if not found.  Raises exception on error.

\paragraph{\code{put}}
\index{put!Ruby API}
\begin{ccode}
Client :: put(spacename, key, attributes)
\end{ccode}
\funcdesc Store an object in space "space" under key "key".

The object will be created if it does not exist.  In the event that the object
already exists, the attributes specified by \texttt{attrs} will overwritten with
their respective values.  Any values not specified will be preserved.  In the
event that the put creates a new object, attributes not specified by
\texttt{attrs} will be initialized to their default values.

\noindent\textbf{Cost:}  Approximately one traversal of the value-dependent
chain.


\noindent\textbf{Consistency:}  Linearizable



\noindent\textbf{Parameters:}
\begin{description}[labelindent=\widthof{{\code{attributes}}},leftmargin=*,noitemsep,nolistsep,align=right]
\item[\code{spacename}] The name of the space as a string or symbol.
\item[\code{key}] The key for the operation as a Ruby type
\item[\code{attributes}] A hash specifying attributes to modify and their respective values.
\end{description}

\noindent\textbf{Returns:}
True.  Raises exception on error.

\paragraph{\code{cond\_put}}
\index{cond\_put!Ruby API}
\begin{ccode}
Client :: cond_put(spacename, key, predicates, attributes)
\end{ccode}
\funcdesc Conditionally update an the object stored under \code{key} in \code{space}.
Existing values will be overwitten with the values specified by \code{attrs}.
Values not specified by \code{attrs} will remain unchanged.
This operation requires a pre-existing object in order to complete successfully.
If no object exists, the operation will fail with \code{NOTFOUND}.


This operation will succeed if and only if the predicates specified by
\code{checks} hold on the pre-existing object.  If any of the predicates are not
true for the existing object, then the operation will have no effect and fail
with \code{CMPFAIL}.

All checks are atomic with the write.  HyperDex guarantees that no other
operation will come between validating the checks, and writing the new version
of the object..



\noindent\textbf{Parameters:}
\begin{description}[labelindent=\widthof{{\code{predicates}}},leftmargin=*,noitemsep,nolistsep,align=right]
\item[\code{spacename}] The name of the space as a string or symbol.
\item[\code{key}] The key for the operation as a Ruby type
\item[\code{predicates}] A hash of predicates to check against.
\item[\code{attributes}] A hash specifying attributes to modify and their respective values.
\end{description}

\noindent\textbf{Returns:}
True if predicate, False if not predicate.  Raises exception on error.

\paragraph{\code{put\_if\_not\_exist}}
\index{put\_if\_not\_exist!Ruby API}
\begin{ccode}
Client :: put_if_not_exist(spacename, key, attributes)
\end{ccode}
\funcdesc Store or object under \code{key} in \code{space} if and only if the operation
creates a new object.  The object's attributes will be set to the values
specified by \code{attrs}; any attributes not specified by \code{attrs} will be
initialized to their defaults.  If the object exists, the operation will fail
with \code{CMPFAIL}.


\noindent\textbf{Parameters:}
\begin{description}[labelindent=\widthof{{\code{attributes}}},leftmargin=*,noitemsep,nolistsep,align=right]
\item[\code{spacename}] The name of the space as a string or symbol.
\item[\code{key}] The key for the operation as a Ruby type
\item[\code{attributes}] A hash specifying attributes to modify and their respective values.
\end{description}

\noindent\textbf{Returns:}
True.  Raises exception on error.

\paragraph{\code{del}}
\index{del!Ruby API}
\begin{ccode}
Client :: del(spacename, key)
\end{ccode}
\funcdesc Delete \code{key} from \code{space}.
If no object exists, the operation will fail with \code{NOTFOUND}.



\noindent\textbf{Parameters:}
\begin{description}[labelindent=\widthof{{\code{spacename}}},leftmargin=*,noitemsep,nolistsep,align=right]
\item[\code{spacename}] The name of the space as a string or symbol.
\item[\code{key}] The key for the operation as a Ruby type
\end{description}

\noindent\textbf{Returns:}
True.  Raises exception on error.

\paragraph{\code{cond\_del}}
\index{cond\_del!Ruby API}
\begin{ccode}
Client :: cond_del(spacename, key, predicates)
\end{ccode}
\funcdesc XXX


\noindent\textbf{Parameters:}
\begin{description}[labelindent=\widthof{{\code{predicates}}},leftmargin=*,noitemsep,nolistsep,align=right]
\item[\code{spacename}] The name of the space as a string or symbol.
\item[\code{key}] The key for the operation as a Ruby type
\item[\code{predicates}] A hash of predicates to check against.
\end{description}

\noindent\textbf{Returns:}
True if predicate, False if not predicate.  Raises exception on error.

\paragraph{\code{atomic\_add}}
\index{atomic\_add!Ruby API}
\begin{ccode}
Client :: atomic_add(spacename, key, attributes)
\end{ccode}
\funcdesc Add the specified number to the existing value for each attribute.
This operation requires a pre-existing object in order to complete successfully.
If no object exists, the operation will fail with \code{NOTFOUND}.



\noindent\textbf{Parameters:}
\begin{description}[labelindent=\widthof{{\code{attributes}}},leftmargin=*,noitemsep,nolistsep,align=right]
\item[\code{spacename}] The name of the space as a string or symbol.
\item[\code{key}] The key for the operation as a Ruby type
\item[\code{attributes}] A hash specifying attributes to modify and their respective values.
\end{description}

\noindent\textbf{Returns:}
True.  Raises exception on error.

\paragraph{\code{cond\_atomic\_add}}
\index{cond\_atomic\_add!Ruby API}
\begin{ccode}
Client :: cond_atomic_add(spacename, key, predicates, attributes)
\end{ccode}
\funcdesc Conditionally add the specified number to the existing value for each attribute.

The operation will modify the object if and only if all \texttt{checks} are true
for the latest version of the object.  This test is atomic with the write.  If
the object does not exist, the checks will fail.

\noindent\textbf{Cost:}  Approximately one traversal of the value-dependent
chain.


\noindent\textbf{Consistency:}  Linearizable



\noindent\textbf{Parameters:}
\begin{description}[labelindent=\widthof{{\code{predicates}}},leftmargin=*,noitemsep,nolistsep,align=right]
\item[\code{spacename}] The name of the space as a string or symbol.
\item[\code{key}] The key for the operation as a Ruby type
\item[\code{predicates}] A hash of predicates to check against.
\item[\code{attributes}] A hash specifying attributes to modify and their respective values.
\end{description}

\noindent\textbf{Returns:}
True if predicate, False if not predicate.  Raises exception on error.

\paragraph{\code{atomic\_sub}}
\index{atomic\_sub!Ruby API}
\begin{ccode}
Client :: atomic_sub(spacename, key, attributes)
\end{ccode}
\funcdesc Subtract the specified number from the existing value for each attribute.
This operation requires a pre-existing object in order to complete successfully.
If no object exists, the operation will fail with \code{NOTFOUND}.



\noindent\textbf{Parameters:}
\begin{description}[labelindent=\widthof{{\code{attributes}}},leftmargin=*,noitemsep,nolistsep,align=right]
\item[\code{spacename}] The name of the space as a string or symbol.
\item[\code{key}] The key for the operation as a Ruby type
\item[\code{attributes}] A hash specifying attributes to modify and their respective values.
\end{description}

\noindent\textbf{Returns:}
True.  Raises exception on error.

\paragraph{\code{cond\_atomic\_sub}}
\index{cond\_atomic\_sub!Ruby API}
\begin{ccode}
Client :: cond_atomic_sub(spacename, key, predicates, attributes)
\end{ccode}
\funcdesc Subtract the specified number from the existing value for each attribute if and
only if the \code{checks} hold on the object.
This operation requires a pre-existing object in order to complete successfully.
If no object exists, the operation will fail with \code{NOTFOUND}.


This operation will succeed if and only if the predicates specified by
\code{checks} hold on the pre-existing object.  If any of the predicates are not
true for the existing object, then the operation will have no effect and fail
with \code{CMPFAIL}.

All checks are atomic with the write.  HyperDex guarantees that no other
operation will come between validating the checks, and writing the new version
of the object..



\noindent\textbf{Parameters:}
\begin{description}[labelindent=\widthof{{\code{predicates}}},leftmargin=*,noitemsep,nolistsep,align=right]
\item[\code{spacename}] The name of the space as a string or symbol.
\item[\code{key}] The key for the operation as a Ruby type
\item[\code{predicates}] A hash of predicates to check against.
\item[\code{attributes}] A hash specifying attributes to modify and their respective values.
\end{description}

\noindent\textbf{Returns:}
True if predicate, False if not predicate.  Raises exception on error.

\paragraph{\code{atomic\_mul}}
\index{atomic\_mul!Ruby API}
\begin{ccode}
Client :: atomic_mul(spacename, key, attributes)
\end{ccode}
\funcdesc Multiply the existing value by the number specified for each attribute.

The multiplication is atomic with the write.  If the object does not exist, the
operation will fail.

\noindent\textbf{Cost:}  Approximately one traversal of the value-dependent
chain.


\noindent\textbf{Consistency:}  Linearizable



\noindent\textbf{Parameters:}
\begin{description}[labelindent=\widthof{{\code{attributes}}},leftmargin=*,noitemsep,nolistsep,align=right]
\item[\code{spacename}] The name of the space as a string or symbol.
\item[\code{key}] The key for the operation as a Ruby type
\item[\code{attributes}] A hash specifying attributes to modify and their respective values.
\end{description}

\noindent\textbf{Returns:}
True.  Raises exception on error.

\paragraph{\code{cond\_atomic\_mul}}
\index{cond\_atomic\_mul!Ruby API}
\begin{ccode}
Client :: cond_atomic_mul(spacename, key, predicates, attributes)
\end{ccode}
\funcdesc Multiply the existing value by the specified number for each attribute if and
only if the \code{checks} hold on the object.
This operation requires a pre-existing object in order to complete successfully.
If no object exists, the operation will fail with \code{NOTFOUND}.


This operation will succeed if and only if the predicates specified by
\code{checks} hold on the pre-existing object.  If any of the predicates are not
true for the existing object, then the operation will have no effect and fail
with \code{CMPFAIL}.

All checks are atomic with the write.  HyperDex guarantees that no other
operation will come between validating the checks, and writing the new version
of the object..



\noindent\textbf{Parameters:}
\begin{description}[labelindent=\widthof{{\code{predicates}}},leftmargin=*,noitemsep,nolistsep,align=right]
\item[\code{spacename}] The name of the space as a string or symbol.
\item[\code{key}] The key for the operation as a Ruby type
\item[\code{predicates}] A hash of predicates to check against.
\item[\code{attributes}] A hash specifying attributes to modify and their respective values.
\end{description}

\noindent\textbf{Returns:}
True if predicate, False if not predicate.  Raises exception on error.

\paragraph{\code{atomic\_div}}
\index{atomic\_div!Ruby API}
\begin{ccode}
Client :: atomic_div(spacename, key, attributes)
\end{ccode}
\funcdesc Divide the existing value by the specified number for each attribute.
This operation requires a pre-existing object in order to complete successfully.
If no object exists, the operation will fail with \code{NOTFOUND}.



\noindent\textbf{Parameters:}
\begin{description}[labelindent=\widthof{{\code{attributes}}},leftmargin=*,noitemsep,nolistsep,align=right]
\item[\code{spacename}] The name of the space as a string or symbol.
\item[\code{key}] The key for the operation as a Ruby type
\item[\code{attributes}] A hash specifying attributes to modify and their respective values.
\end{description}

\noindent\textbf{Returns:}
True.  Raises exception on error.

\paragraph{\code{cond\_atomic\_div}}
\index{cond\_atomic\_div!Ruby API}
\begin{ccode}
Client :: cond_atomic_div(spacename, key, predicates, attributes)
\end{ccode}
\funcdesc Divide the existing value by the specified number for each attribute if and only
if the \code{checks} hold on the object.
This operation requires a pre-existing object in order to complete successfully.
If no object exists, the operation will fail with \code{NOTFOUND}.


This operation will succeed if and only if the predicates specified by
\code{checks} hold on the pre-existing object.  If any of the predicates are not
true for the existing object, then the operation will have no effect and fail
with \code{CMPFAIL}.

All checks are atomic with the write.  HyperDex guarantees that no other
operation will come between validating the checks, and writing the new version
of the object..



\noindent\textbf{Parameters:}
\begin{description}[labelindent=\widthof{{\code{predicates}}},leftmargin=*,noitemsep,nolistsep,align=right]
\item[\code{spacename}] The name of the space as a string or symbol.
\item[\code{key}] The key for the operation as a Ruby type
\item[\code{predicates}] A hash of predicates to check against.
\item[\code{attributes}] A hash specifying attributes to modify and their respective values.
\end{description}

\noindent\textbf{Returns:}
True if predicate, False if not predicate.  Raises exception on error.

\paragraph{\code{atomic\_mod}}
\index{atomic\_mod!Ruby API}
\begin{ccode}
Client :: atomic_mod(spacename, key, attributes)
\end{ccode}
\funcdesc XXX


\noindent\textbf{Parameters:}
\begin{description}[labelindent=\widthof{{\code{attributes}}},leftmargin=*,noitemsep,nolistsep,align=right]
\item[\code{spacename}] The name of the space as a string or symbol.
\item[\code{key}] The key for the operation as a Ruby type
\item[\code{attributes}] A hash specifying attributes to modify and their respective values.
\end{description}

\noindent\textbf{Returns:}
True.  Raises exception on error.

\paragraph{\code{cond\_atomic\_mod}}
\index{cond\_atomic\_mod!Ruby API}
\begin{ccode}
Client :: cond_atomic_mod(spacename, key, predicates, attributes)
\end{ccode}
\funcdesc XXX


\noindent\textbf{Parameters:}
\begin{description}[labelindent=\widthof{{\code{predicates}}},leftmargin=*,noitemsep,nolistsep,align=right]
\item[\code{spacename}] The name of the space as a string or symbol.
\item[\code{key}] The key for the operation as a Ruby type
\item[\code{predicates}] A hash of predicates to check against.
\item[\code{attributes}] A hash specifying attributes to modify and their respective values.
\end{description}

\noindent\textbf{Returns:}
True if predicate, False if not predicate.  Raises exception on error.

\paragraph{\code{atomic\_and}}
\index{atomic\_and!Ruby API}
\begin{ccode}
Client :: atomic_and(spacename, key, attributes)
\end{ccode}
\funcdesc XXX


\noindent\textbf{Parameters:}
\begin{description}[labelindent=\widthof{{\code{attributes}}},leftmargin=*,noitemsep,nolistsep,align=right]
\item[\code{spacename}] The name of the space as a string or symbol.
\item[\code{key}] The key for the operation as a Ruby type
\item[\code{attributes}] A hash specifying attributes to modify and their respective values.
\end{description}

\noindent\textbf{Returns:}
True.  Raises exception on error.

\paragraph{\code{cond\_atomic\_and}}
\index{cond\_atomic\_and!Ruby API}
\begin{ccode}
Client :: cond_atomic_and(spacename, key, predicates, attributes)
\end{ccode}
\funcdesc Store the bitwise AND of the existing value and the specified number for
each attribute if and only if the \code{checks} hold on the object.
This operation requires a pre-existing object in order to complete successfully.
If no object exists, the operation will fail with \code{NOTFOUND}.


This operation will succeed if and only if the predicates specified by
\code{checks} hold on the pre-existing object.  If any of the predicates are not
true for the existing object, then the operation will have no effect and fail
with \code{CMPFAIL}.

All checks are atomic with the write.  HyperDex guarantees that no other
operation will come between validating the checks, and writing the new version
of the object..



\noindent\textbf{Parameters:}
\begin{description}[labelindent=\widthof{{\code{predicates}}},leftmargin=*,noitemsep,nolistsep,align=right]
\item[\code{spacename}] The name of the space as a string or symbol.
\item[\code{key}] The key for the operation as a Ruby type
\item[\code{predicates}] A hash of predicates to check against.
\item[\code{attributes}] A hash specifying attributes to modify and their respective values.
\end{description}

\noindent\textbf{Returns:}
True if predicate, False if not predicate.  Raises exception on error.

\paragraph{\code{atomic\_or}}
\index{atomic\_or!Ruby API}
\begin{ccode}
Client :: atomic_or(spacename, key, attributes)
\end{ccode}
\funcdesc XXX


\noindent\textbf{Parameters:}
\begin{description}[labelindent=\widthof{{\code{attributes}}},leftmargin=*,noitemsep,nolistsep,align=right]
\item[\code{spacename}] The name of the space as a string or symbol.
\item[\code{key}] The key for the operation as a Ruby type
\item[\code{attributes}] A hash specifying attributes to modify and their respective values.
\end{description}

\noindent\textbf{Returns:}
True.  Raises exception on error.

\paragraph{\code{cond\_atomic\_or}}
\index{cond\_atomic\_or!Ruby API}
\begin{ccode}
Client :: cond_atomic_or(spacename, key, predicates, attributes)
\end{ccode}
\funcdesc XXX


\noindent\textbf{Parameters:}
\begin{description}[labelindent=\widthof{{\code{predicates}}},leftmargin=*,noitemsep,nolistsep,align=right]
\item[\code{spacename}] The name of the space as a string or symbol.
\item[\code{key}] The key for the operation as a Ruby type
\item[\code{predicates}] A hash of predicates to check against.
\item[\code{attributes}] A hash specifying attributes to modify and their respective values.
\end{description}

\noindent\textbf{Returns:}
True if predicate, False if not predicate.  Raises exception on error.

\paragraph{\code{atomic\_xor}}
\index{atomic\_xor!Ruby API}
\begin{ccode}
Client :: atomic_xor(spacename, key, attributes)
\end{ccode}
\funcdesc XXX


\noindent\textbf{Parameters:}
\begin{description}[labelindent=\widthof{{\code{attributes}}},leftmargin=*,noitemsep,nolistsep,align=right]
\item[\code{spacename}] The name of the space as a string or symbol.
\item[\code{key}] The key for the operation as a Ruby type
\item[\code{attributes}] A hash specifying attributes to modify and their respective values.
\end{description}

\noindent\textbf{Returns:}
True.  Raises exception on error.

\paragraph{\code{cond\_atomic\_xor}}
\index{cond\_atomic\_xor!Ruby API}
\begin{ccode}
Client :: cond_atomic_xor(spacename, key, predicates, attributes)
\end{ccode}
\funcdesc XXX


\noindent\textbf{Parameters:}
\begin{description}[labelindent=\widthof{{\code{predicates}}},leftmargin=*,noitemsep,nolistsep,align=right]
\item[\code{spacename}] The name of the space as a string or symbol.
\item[\code{key}] The key for the operation as a Ruby type
\item[\code{predicates}] A hash of predicates to check against.
\item[\code{attributes}] A hash specifying attributes to modify and their respective values.
\end{description}

\noindent\textbf{Returns:}
True if predicate, False if not predicate.  Raises exception on error.

\paragraph{\code{string\_prepend}}
\index{string\_prepend!Ruby API}
\begin{ccode}
Client :: string_prepend(spacename, key, attributes)
\end{ccode}
\funcdesc Prepend the specified string to the existing value for each attribute.
This operation requires a pre-existing object in order to complete successfully.
If no object exists, the operation will fail with \code{NOTFOUND}.



\noindent\textbf{Parameters:}
\begin{description}[labelindent=\widthof{{\code{attributes}}},leftmargin=*,noitemsep,nolistsep,align=right]
\item[\code{spacename}] The name of the space as a string or symbol.
\item[\code{key}] The key for the operation as a Ruby type
\item[\code{attributes}] A hash specifying attributes to modify and their respective values.
\end{description}

\noindent\textbf{Returns:}
True.  Raises exception on error.

\paragraph{\code{cond\_string\_prepend}}
\index{cond\_string\_prepend!Ruby API}
\begin{ccode}
Client :: cond_string_prepend(spacename, key, predicates, attributes)
\end{ccode}
\funcdesc XXX


\noindent\textbf{Parameters:}
\begin{description}[labelindent=\widthof{{\code{predicates}}},leftmargin=*,noitemsep,nolistsep,align=right]
\item[\code{spacename}] The name of the space as a string or symbol.
\item[\code{key}] The key for the operation as a Ruby type
\item[\code{predicates}] A hash of predicates to check against.
\item[\code{attributes}] A hash specifying attributes to modify and their respective values.
\end{description}

\noindent\textbf{Returns:}
True if predicate, False if not predicate.  Raises exception on error.

\paragraph{\code{string\_append}}
\index{string\_append!Ruby API}
\begin{ccode}
Client :: string_append(spacename, key, attributes)
\end{ccode}
\funcdesc XXX


\noindent\textbf{Parameters:}
\begin{description}[labelindent=\widthof{{\code{attributes}}},leftmargin=*,noitemsep,nolistsep,align=right]
\item[\code{spacename}] The name of the space as a string or symbol.
\item[\code{key}] The key for the operation as a Ruby type
\item[\code{attributes}] A hash specifying attributes to modify and their respective values.
\end{description}

\noindent\textbf{Returns:}
True.  Raises exception on error.

\paragraph{\code{cond\_string\_append}}
\index{cond\_string\_append!Ruby API}
\begin{ccode}
Client :: cond_string_append(spacename, key, predicates, attributes)
\end{ccode}
\funcdesc XXX


\noindent\textbf{Parameters:}
\begin{description}[labelindent=\widthof{{\code{predicates}}},leftmargin=*,noitemsep,nolistsep,align=right]
\item[\code{spacename}] The name of the space as a string or symbol.
\item[\code{key}] The key for the operation as a Ruby type
\item[\code{predicates}] A hash of predicates to check against.
\item[\code{attributes}] A hash specifying attributes to modify and their respective values.
\end{description}

\noindent\textbf{Returns:}
True if predicate, False if not predicate.  Raises exception on error.

\paragraph{\code{list\_lpush}}
\index{list\_lpush!Ruby API}
\begin{ccode}
Client :: list_lpush(spacename, key, attributes)
\end{ccode}
\funcdesc XXX


\noindent\textbf{Parameters:}
\begin{description}[labelindent=\widthof{{\code{attributes}}},leftmargin=*,noitemsep,nolistsep,align=right]
\item[\code{spacename}] The name of the space as a string or symbol.
\item[\code{key}] The key for the operation as a Ruby type
\item[\code{attributes}] A hash specifying attributes to modify and their respective values.
\end{description}

\noindent\textbf{Returns:}
True.  Raises exception on error.

\paragraph{\code{cond\_list\_lpush}}
\index{cond\_list\_lpush!Ruby API}
\begin{ccode}
Client :: cond_list_lpush(spacename, key, predicates, attributes)
\end{ccode}
\funcdesc Push the specified value onto the front of the list for each attribute if and
only if \code{checks} hold on the object.
This operation requires a pre-existing object in order to complete successfully.
If no object exists, the operation will fail with \code{NOTFOUND}.


This operation will succeed if and only if the predicates specified by
\code{checks} hold on the pre-existing object.  If any of the predicates are not
true for the existing object, then the operation will have no effect and fail
with \code{CMPFAIL}.

All checks are atomic with the write.  HyperDex guarantees that no other
operation will come between validating the checks, and writing the new version
of the object..



\noindent\textbf{Parameters:}
\begin{description}[labelindent=\widthof{{\code{predicates}}},leftmargin=*,noitemsep,nolistsep,align=right]
\item[\code{spacename}] The name of the space as a string or symbol.
\item[\code{key}] The key for the operation as a Ruby type
\item[\code{predicates}] A hash of predicates to check against.
\item[\code{attributes}] A hash specifying attributes to modify and their respective values.
\end{description}

\noindent\textbf{Returns:}
True if predicate, False if not predicate.  Raises exception on error.

\paragraph{\code{list\_rpush}}
\index{list\_rpush!Ruby API}
\begin{ccode}
Client :: list_rpush(spacename, key, attributes)
\end{ccode}
\funcdesc XXX


\noindent\textbf{Parameters:}
\begin{description}[labelindent=\widthof{{\code{attributes}}},leftmargin=*,noitemsep,nolistsep,align=right]
\item[\code{spacename}] The name of the space as a string or symbol.
\item[\code{key}] The key for the operation as a Ruby type
\item[\code{attributes}] A hash specifying attributes to modify and their respective values.
\end{description}

\noindent\textbf{Returns:}
True.  Raises exception on error.

\paragraph{\code{cond\_list\_rpush}}
\index{cond\_list\_rpush!Ruby API}
\begin{ccode}
Client :: cond_list_rpush(spacename, key, predicates, attributes)
\end{ccode}
\funcdesc Push the specified value onto the back of the list for each attribute if and
only if the \code{checks} hold on the object.
This operation requires a pre-existing object in order to complete successfully.
If no object exists, the operation will fail with \code{NOTFOUND}.


This operation will succeed if and only if the predicates specified by
\code{checks} hold on the pre-existing object.  If any of the predicates are not
true for the existing object, then the operation will have no effect and fail
with \code{CMPFAIL}.

All checks are atomic with the write.  HyperDex guarantees that no other
operation will come between validating the checks, and writing the new version
of the object..



\noindent\textbf{Parameters:}
\begin{description}[labelindent=\widthof{{\code{predicates}}},leftmargin=*,noitemsep,nolistsep,align=right]
\item[\code{spacename}] The name of the space as a string or symbol.
\item[\code{key}] The key for the operation as a Ruby type
\item[\code{predicates}] A hash of predicates to check against.
\item[\code{attributes}] A hash specifying attributes to modify and their respective values.
\end{description}

\noindent\textbf{Returns:}
True if predicate, False if not predicate.  Raises exception on error.

\paragraph{\code{set\_add}}
\index{set\_add!Ruby API}
\begin{ccode}
Client :: set_add(spacename, key, attributes)
\end{ccode}
\funcdesc XXX


\noindent\textbf{Parameters:}
\begin{description}[labelindent=\widthof{{\code{attributes}}},leftmargin=*,noitemsep,nolistsep,align=right]
\item[\code{spacename}] The name of the space as a string or symbol.
\item[\code{key}] The key for the operation as a Ruby type
\item[\code{attributes}] A hash specifying attributes to modify and their respective values.
\end{description}

\noindent\textbf{Returns:}
True.  Raises exception on error.

\paragraph{\code{cond\_set\_add}}
\index{cond\_set\_add!Ruby API}
\begin{ccode}
Client :: cond_set_add(spacename, key, predicates, attributes)
\end{ccode}
\funcdesc Add the specified value to the set for each attribute if and only if the
\code{checks} hold on the object.
This operation requires a pre-existing object in order to complete successfully.
If no object exists, the operation will fail with \code{NOTFOUND}.


This operation will succeed if and only if the predicates specified by
\code{checks} hold on the pre-existing object.  If any of the predicates are not
true for the existing object, then the operation will have no effect and fail
with \code{CMPFAIL}.

All checks are atomic with the write.  HyperDex guarantees that no other
operation will come between validating the checks, and writing the new version
of the object..



\noindent\textbf{Parameters:}
\begin{description}[labelindent=\widthof{{\code{predicates}}},leftmargin=*,noitemsep,nolistsep,align=right]
\item[\code{spacename}] The name of the space as a string or symbol.
\item[\code{key}] The key for the operation as a Ruby type
\item[\code{predicates}] A hash of predicates to check against.
\item[\code{attributes}] A hash specifying attributes to modify and their respective values.
\end{description}

\noindent\textbf{Returns:}
True if predicate, False if not predicate.  Raises exception on error.

\paragraph{\code{set\_remove}}
\index{set\_remove!Ruby API}
\begin{ccode}
Client :: set_remove(spacename, key, attributes)
\end{ccode}
\funcdesc Remove the specified value from the set.  If the value is not contained within
the set, this operation will do nothing.
This operation requires a pre-existing object in order to complete successfully.
If no object exists, the operation will fail with \code{NOTFOUND}.



\noindent\textbf{Parameters:}
\begin{description}[labelindent=\widthof{{\code{attributes}}},leftmargin=*,noitemsep,nolistsep,align=right]
\item[\code{spacename}] The name of the space as a string or symbol.
\item[\code{key}] The key for the operation as a Ruby type
\item[\code{attributes}] A hash specifying attributes to modify and their respective values.
\end{description}

\noindent\textbf{Returns:}
True.  Raises exception on error.

\paragraph{\code{cond\_set\_remove}}
\index{cond\_set\_remove!Ruby API}
\begin{ccode}
Client :: cond_set_remove(spacename, key, predicates, attributes)
\end{ccode}
\funcdesc XXX


\noindent\textbf{Parameters:}
\begin{description}[labelindent=\widthof{{\code{predicates}}},leftmargin=*,noitemsep,nolistsep,align=right]
\item[\code{spacename}] The name of the space as a string or symbol.
\item[\code{key}] The key for the operation as a Ruby type
\item[\code{predicates}] A hash of predicates to check against.
\item[\code{attributes}] A hash specifying attributes to modify and their respective values.
\end{description}

\noindent\textbf{Returns:}
True if predicate, False if not predicate.  Raises exception on error.

\paragraph{\code{set\_intersect}}
\index{set\_intersect!Ruby API}
\begin{ccode}
Client :: set_intersect(spacename, key, attributes)
\end{ccode}
\funcdesc XXX


\noindent\textbf{Parameters:}
\begin{description}[labelindent=\widthof{{\code{attributes}}},leftmargin=*,noitemsep,nolistsep,align=right]
\item[\code{spacename}] The name of the space as a string or symbol.
\item[\code{key}] The key for the operation as a Ruby type
\item[\code{attributes}] A hash specifying attributes to modify and their respective values.
\end{description}

\noindent\textbf{Returns:}
True.  Raises exception on error.

\paragraph{\code{cond\_set\_intersect}}
\index{cond\_set\_intersect!Ruby API}
\begin{ccode}
Client :: cond_set_intersect(spacename, key, predicates, attributes)
\end{ccode}
\funcdesc XXX


\noindent\textbf{Parameters:}
\begin{description}[labelindent=\widthof{{\code{predicates}}},leftmargin=*,noitemsep,nolistsep,align=right]
\item[\code{spacename}] The name of the space as a string or symbol.
\item[\code{key}] The key for the operation as a Ruby type
\item[\code{predicates}] A hash of predicates to check against.
\item[\code{attributes}] A hash specifying attributes to modify and their respective values.
\end{description}

\noindent\textbf{Returns:}
True if predicate, False if not predicate.  Raises exception on error.

\paragraph{\code{set\_union}}
\index{set\_union!Ruby API}
\begin{ccode}
Client :: set_union(spacename, key, attributes)
\end{ccode}
\funcdesc XXX


\noindent\textbf{Parameters:}
\begin{description}[labelindent=\widthof{{\code{attributes}}},leftmargin=*,noitemsep,nolistsep,align=right]
\item[\code{spacename}] The name of the space as a string or symbol.
\item[\code{key}] The key for the operation as a Ruby type
\item[\code{attributes}] A hash specifying attributes to modify and their respective values.
\end{description}

\noindent\textbf{Returns:}
True.  Raises exception on error.

\paragraph{\code{cond\_set\_union}}
\index{cond\_set\_union!Ruby API}
\begin{ccode}
Client :: cond_set_union(spacename, key, predicates, attributes)
\end{ccode}
\funcdesc XXX


\noindent\textbf{Parameters:}
\begin{description}[labelindent=\widthof{{\code{predicates}}},leftmargin=*,noitemsep,nolistsep,align=right]
\item[\code{spacename}] The name of the space as a string or symbol.
\item[\code{key}] The key for the operation as a Ruby type
\item[\code{predicates}] A hash of predicates to check against.
\item[\code{attributes}] A hash specifying attributes to modify and their respective values.
\end{description}

\noindent\textbf{Returns:}
True if predicate, False if not predicate.  Raises exception on error.

\paragraph{\code{map\_add}}
\index{map\_add!Ruby API}
\begin{ccode}
Client :: map_add(spacename, key, mapattributes)
\end{ccode}
\funcdesc XXX


\noindent\textbf{Parameters:}
\begin{description}[labelindent=\widthof{{\code{mapattributes}}},leftmargin=*,noitemsep,nolistsep,align=right]
\item[\code{spacename}] The name of the space as a string or symbol.
\item[\code{key}] The key for the operation as a Ruby type
\item[\code{mapattributes}] A hash specifying map attributes to modify and their respective key/values.
\end{description}

\noindent\textbf{Returns:}
True.  Raises exception on error.

\paragraph{\code{cond\_map\_add}}
\index{cond\_map\_add!Ruby API}
\begin{ccode}
Client :: cond_map_add(spacename, key, predicates, mapattributes)
\end{ccode}
\funcdesc XXX


\noindent\textbf{Parameters:}
\begin{description}[labelindent=\widthof{{\code{mapattributes}}},leftmargin=*,noitemsep,nolistsep,align=right]
\item[\code{spacename}] The name of the space as a string or symbol.
\item[\code{key}] The key for the operation as a Ruby type
\item[\code{predicates}] A hash of predicates to check against.
\item[\code{mapattributes}] A hash specifying map attributes to modify and their respective key/values.
\end{description}

\noindent\textbf{Returns:}
True if predicate, False if not predicate.  Raises exception on error.

\paragraph{\code{map\_remove}}
\index{map\_remove!Ruby API}
\begin{ccode}
Client :: map_remove(spacename, key, attributes)
\end{ccode}
\funcdesc XXX


\noindent\textbf{Parameters:}
\begin{description}[labelindent=\widthof{{\code{attributes}}},leftmargin=*,noitemsep,nolistsep,align=right]
\item[\code{spacename}] The name of the space as a string or symbol.
\item[\code{key}] The key for the operation as a Ruby type
\item[\code{attributes}] A hash specifying attributes to modify and their respective values.
\end{description}

\noindent\textbf{Returns:}
True.  Raises exception on error.

\paragraph{\code{cond\_map\_remove}}
\index{cond\_map\_remove!Ruby API}
\begin{ccode}
Client :: cond_map_remove(spacename, key, predicates, attributes)
\end{ccode}
\funcdesc XXX


\noindent\textbf{Parameters:}
\begin{description}[labelindent=\widthof{{\code{predicates}}},leftmargin=*,noitemsep,nolistsep,align=right]
\item[\code{spacename}] The name of the space as a string or symbol.
\item[\code{key}] The key for the operation as a Ruby type
\item[\code{predicates}] A hash of predicates to check against.
\item[\code{attributes}] A hash specifying attributes to modify and their respective values.
\end{description}

\noindent\textbf{Returns:}
True if predicate, False if not predicate.  Raises exception on error.

\paragraph{\code{map\_atomic\_add}}
\index{map\_atomic\_add!Ruby API}
\begin{ccode}
Client :: map_atomic_add(spacename, key, mapattributes)
\end{ccode}
\funcdesc Add the specified number to the value of a key-value pair within each map.
This operation requires a pre-existing object in order to complete successfully.
If no object exists, the operation will fail with \code{NOTFOUND}.



\noindent\textbf{Parameters:}
\begin{description}[labelindent=\widthof{{\code{mapattributes}}},leftmargin=*,noitemsep,nolistsep,align=right]
\item[\code{spacename}] The name of the space as a string or symbol.
\item[\code{key}] The key for the operation as a Ruby type
\item[\code{mapattributes}] A hash specifying map attributes to modify and their respective key/values.
\end{description}

\noindent\textbf{Returns:}
True.  Raises exception on error.

\paragraph{\code{cond\_map\_atomic\_add}}
\index{cond\_map\_atomic\_add!Ruby API}
\begin{ccode}
Client :: cond_map_atomic_add(spacename, key, predicates, mapattributes)
\end{ccode}
\funcdesc Add the specified number to the value of a key-value pair within each map if and
only if the \code{checks} hold on the object.
This operation requires a pre-existing object in order to complete successfully.
If no object exists, the operation will fail with \code{NOTFOUND}.


This operation will succeed if and only if the predicates specified by
\code{checks} hold on the pre-existing object.  If any of the predicates are not
true for the existing object, then the operation will have no effect and fail
with \code{CMPFAIL}.

All checks are atomic with the write.  HyperDex guarantees that no other
operation will come between validating the checks, and writing the new version
of the object..



\noindent\textbf{Parameters:}
\begin{description}[labelindent=\widthof{{\code{mapattributes}}},leftmargin=*,noitemsep,nolistsep,align=right]
\item[\code{spacename}] The name of the space as a string or symbol.
\item[\code{key}] The key for the operation as a Ruby type
\item[\code{predicates}] A hash of predicates to check against.
\item[\code{mapattributes}] A hash specifying map attributes to modify and their respective key/values.
\end{description}

\noindent\textbf{Returns:}
True if predicate, False if not predicate.  Raises exception on error.

\paragraph{\code{map\_atomic\_sub}}
\index{map\_atomic\_sub!Ruby API}
\begin{ccode}
Client :: map_atomic_sub(spacename, key, mapattributes)
\end{ccode}
\funcdesc XXX


\noindent\textbf{Parameters:}
\begin{description}[labelindent=\widthof{{\code{mapattributes}}},leftmargin=*,noitemsep,nolistsep,align=right]
\item[\code{spacename}] The name of the space as a string or symbol.
\item[\code{key}] The key for the operation as a Ruby type
\item[\code{mapattributes}] A hash specifying map attributes to modify and their respective key/values.
\end{description}

\noindent\textbf{Returns:}
True.  Raises exception on error.

\paragraph{\code{cond\_map\_atomic\_sub}}
\index{cond\_map\_atomic\_sub!Ruby API}
\begin{ccode}
Client :: cond_map_atomic_sub(spacename, key, predicates, mapattributes)
\end{ccode}
\funcdesc XXX


\noindent\textbf{Parameters:}
\begin{description}[labelindent=\widthof{{\code{mapattributes}}},leftmargin=*,noitemsep,nolistsep,align=right]
\item[\code{spacename}] The name of the space as a string or symbol.
\item[\code{key}] The key for the operation as a Ruby type
\item[\code{predicates}] A hash of predicates to check against.
\item[\code{mapattributes}] A hash specifying map attributes to modify and their respective key/values.
\end{description}

\noindent\textbf{Returns:}
True if predicate, False if not predicate.  Raises exception on error.

\paragraph{\code{map\_atomic\_mul}}
\index{map\_atomic\_mul!Ruby API}
\begin{ccode}
Client :: map_atomic_mul(spacename, key, mapattributes)
\end{ccode}
\funcdesc XXX


\noindent\textbf{Parameters:}
\begin{description}[labelindent=\widthof{{\code{mapattributes}}},leftmargin=*,noitemsep,nolistsep,align=right]
\item[\code{spacename}] The name of the space as a string or symbol.
\item[\code{key}] The key for the operation as a Ruby type
\item[\code{mapattributes}] A hash specifying map attributes to modify and their respective key/values.
\end{description}

\noindent\textbf{Returns:}
True.  Raises exception on error.

\paragraph{\code{cond\_map\_atomic\_mul}}
\index{cond\_map\_atomic\_mul!Ruby API}
\begin{ccode}
Client :: cond_map_atomic_mul(spacename, key, predicates, mapattributes)
\end{ccode}
\funcdesc XXX


\noindent\textbf{Parameters:}
\begin{description}[labelindent=\widthof{{\code{mapattributes}}},leftmargin=*,noitemsep,nolistsep,align=right]
\item[\code{spacename}] The name of the space as a string or symbol.
\item[\code{key}] The key for the operation as a Ruby type
\item[\code{predicates}] A hash of predicates to check against.
\item[\code{mapattributes}] A hash specifying map attributes to modify and their respective key/values.
\end{description}

\noindent\textbf{Returns:}
True if predicate, False if not predicate.  Raises exception on error.

\paragraph{\code{map\_atomic\_div}}
\index{map\_atomic\_div!Ruby API}
\begin{ccode}
Client :: map_atomic_div(spacename, key, mapattributes)
\end{ccode}
\funcdesc XXX


\noindent\textbf{Parameters:}
\begin{description}[labelindent=\widthof{{\code{mapattributes}}},leftmargin=*,noitemsep,nolistsep,align=right]
\item[\code{spacename}] The name of the space as a string or symbol.
\item[\code{key}] The key for the operation as a Ruby type
\item[\code{mapattributes}] A hash specifying map attributes to modify and their respective key/values.
\end{description}

\noindent\textbf{Returns:}
True.  Raises exception on error.

\paragraph{\code{cond\_map\_atomic\_div}}
\index{cond\_map\_atomic\_div!Ruby API}
\begin{ccode}
Client :: cond_map_atomic_div(spacename, key, predicates, mapattributes)
\end{ccode}
\funcdesc XXX


\noindent\textbf{Parameters:}
\begin{description}[labelindent=\widthof{{\code{mapattributes}}},leftmargin=*,noitemsep,nolistsep,align=right]
\item[\code{spacename}] The name of the space as a string or symbol.
\item[\code{key}] The key for the operation as a Ruby type
\item[\code{predicates}] A hash of predicates to check against.
\item[\code{mapattributes}] A hash specifying map attributes to modify and their respective key/values.
\end{description}

\noindent\textbf{Returns:}
True if predicate, False if not predicate.  Raises exception on error.

\paragraph{\code{map\_atomic\_mod}}
\index{map\_atomic\_mod!Ruby API}
\begin{ccode}
Client :: map_atomic_mod(spacename, key, mapattributes)
\end{ccode}
\funcdesc Store the value of the key-value pair modulo the specified number for each map.
This operation requires a pre-existing object in order to complete successfully.
If no object exists, the operation will fail with \code{NOTFOUND}.



\noindent\textbf{Parameters:}
\begin{description}[labelindent=\widthof{{\code{mapattributes}}},leftmargin=*,noitemsep,nolistsep,align=right]
\item[\code{spacename}] The name of the space as a string or symbol.
\item[\code{key}] The key for the operation as a Ruby type
\item[\code{mapattributes}] A hash specifying map attributes to modify and their respective key/values.
\end{description}

\noindent\textbf{Returns:}
True.  Raises exception on error.

\paragraph{\code{cond\_map\_atomic\_mod}}
\index{cond\_map\_atomic\_mod!Ruby API}
\begin{ccode}
Client :: cond_map_atomic_mod(spacename, key, predicates, mapattributes)
\end{ccode}
\funcdesc XXX


\noindent\textbf{Parameters:}
\begin{description}[labelindent=\widthof{{\code{mapattributes}}},leftmargin=*,noitemsep,nolistsep,align=right]
\item[\code{spacename}] The name of the space as a string or symbol.
\item[\code{key}] The key for the operation as a Ruby type
\item[\code{predicates}] A hash of predicates to check against.
\item[\code{mapattributes}] A hash specifying map attributes to modify and their respective key/values.
\end{description}

\noindent\textbf{Returns:}
True if predicate, False if not predicate.  Raises exception on error.

\paragraph{\code{map\_atomic\_and}}
\index{map\_atomic\_and!Ruby API}
\begin{ccode}
Client :: map_atomic_and(spacename, key, mapattributes)
\end{ccode}
\funcdesc XXX


\noindent\textbf{Parameters:}
\begin{description}[labelindent=\widthof{{\code{mapattributes}}},leftmargin=*,noitemsep,nolistsep,align=right]
\item[\code{spacename}] The name of the space as a string or symbol.
\item[\code{key}] The key for the operation as a Ruby type
\item[\code{mapattributes}] A hash specifying map attributes to modify and their respective key/values.
\end{description}

\noindent\textbf{Returns:}
True.  Raises exception on error.

\paragraph{\code{cond\_map\_atomic\_and}}
\index{cond\_map\_atomic\_and!Ruby API}
\begin{ccode}
Client :: cond_map_atomic_and(spacename, key, predicates, mapattributes)
\end{ccode}
\funcdesc Store the bitwise AND of the value of the key-value pair and the specified
number for each map attribute if and only if the \code{checks} hold on the
object.
This operation requires a pre-existing object in order to complete successfully.
If no object exists, the operation will fail with \code{NOTFOUND}.


This operation will succeed if and only if the predicates specified by
\code{checks} hold on the pre-existing object.  If any of the predicates are not
true for the existing object, then the operation will have no effect and fail
with \code{CMPFAIL}.

All checks are atomic with the write.  HyperDex guarantees that no other
operation will come between validating the checks, and writing the new version
of the object..



\noindent\textbf{Parameters:}
\begin{description}[labelindent=\widthof{{\code{mapattributes}}},leftmargin=*,noitemsep,nolistsep,align=right]
\item[\code{spacename}] The name of the space as a string or symbol.
\item[\code{key}] The key for the operation as a Ruby type
\item[\code{predicates}] A hash of predicates to check against.
\item[\code{mapattributes}] A hash specifying map attributes to modify and their respective key/values.
\end{description}

\noindent\textbf{Returns:}
True if predicate, False if not predicate.  Raises exception on error.

\paragraph{\code{map\_atomic\_or}}
\index{map\_atomic\_or!Ruby API}
\begin{ccode}
Client :: map_atomic_or(spacename, key, mapattributes)
\end{ccode}
\funcdesc XXX


\noindent\textbf{Parameters:}
\begin{description}[labelindent=\widthof{{\code{mapattributes}}},leftmargin=*,noitemsep,nolistsep,align=right]
\item[\code{spacename}] The name of the space as a string or symbol.
\item[\code{key}] The key for the operation as a Ruby type
\item[\code{mapattributes}] A hash specifying map attributes to modify and their respective key/values.
\end{description}

\noindent\textbf{Returns:}
True.  Raises exception on error.

\paragraph{\code{cond\_map\_atomic\_or}}
\index{cond\_map\_atomic\_or!Ruby API}
\begin{ccode}
Client :: cond_map_atomic_or(spacename, key, predicates, mapattributes)
\end{ccode}
\funcdesc Store the bitwise OR of the value of the key-value pair and the specified number
for each map attribute if and only if the \code{checks} hold on the object.
This operation requires a pre-existing object in order to complete successfully.
If no object exists, the operation will fail with \code{NOTFOUND}.


This operation will succeed if and only if the predicates specified by
\code{checks} hold on the pre-existing object.  If any of the predicates are not
true for the existing object, then the operation will have no effect and fail
with \code{CMPFAIL}.

All checks are atomic with the write.  HyperDex guarantees that no other
operation will come between validating the checks, and writing the new version
of the object..



\noindent\textbf{Parameters:}
\begin{description}[labelindent=\widthof{{\code{mapattributes}}},leftmargin=*,noitemsep,nolistsep,align=right]
\item[\code{spacename}] The name of the space as a string or symbol.
\item[\code{key}] The key for the operation as a Ruby type
\item[\code{predicates}] A hash of predicates to check against.
\item[\code{mapattributes}] A hash specifying map attributes to modify and their respective key/values.
\end{description}

\noindent\textbf{Returns:}
True if predicate, False if not predicate.  Raises exception on error.

\paragraph{\code{map\_atomic\_xor}}
\index{map\_atomic\_xor!Ruby API}
\begin{ccode}
Client :: map_atomic_xor(spacename, key, mapattributes)
\end{ccode}
\funcdesc XXX


\noindent\textbf{Parameters:}
\begin{description}[labelindent=\widthof{{\code{mapattributes}}},leftmargin=*,noitemsep,nolistsep,align=right]
\item[\code{spacename}] The name of the space as a string or symbol.
\item[\code{key}] The key for the operation as a Ruby type
\item[\code{mapattributes}] A hash specifying map attributes to modify and their respective key/values.
\end{description}

\noindent\textbf{Returns:}
True.  Raises exception on error.

\paragraph{\code{cond\_map\_atomic\_xor}}
\index{cond\_map\_atomic\_xor!Ruby API}
\begin{ccode}
Client :: cond_map_atomic_xor(spacename, key, predicates, mapattributes)
\end{ccode}
\funcdesc XXX


\noindent\textbf{Parameters:}
\begin{description}[labelindent=\widthof{{\code{mapattributes}}},leftmargin=*,noitemsep,nolistsep,align=right]
\item[\code{spacename}] The name of the space as a string or symbol.
\item[\code{key}] The key for the operation as a Ruby type
\item[\code{predicates}] A hash of predicates to check against.
\item[\code{mapattributes}] A hash specifying map attributes to modify and their respective key/values.
\end{description}

\noindent\textbf{Returns:}
True if predicate, False if not predicate.  Raises exception on error.

\paragraph{\code{map\_string\_prepend}}
\index{map\_string\_prepend!Ruby API}
\begin{ccode}
Client :: map_string_prepend(spacename, key, mapattributes)
\end{ccode}
\funcdesc XXX


\noindent\textbf{Parameters:}
\begin{description}[labelindent=\widthof{{\code{mapattributes}}},leftmargin=*,noitemsep,nolistsep,align=right]
\item[\code{spacename}] The name of the space as a string or symbol.
\item[\code{key}] The key for the operation as a Ruby type
\item[\code{mapattributes}] A hash specifying map attributes to modify and their respective key/values.
\end{description}

\noindent\textbf{Returns:}
True.  Raises exception on error.

\paragraph{\code{cond\_map\_string\_prepend}}
\index{cond\_map\_string\_prepend!Ruby API}
\begin{ccode}
Client :: cond_map_string_prepend(spacename, key, predicates, mapattributes)
\end{ccode}
\funcdesc XXX


\noindent\textbf{Parameters:}
\begin{description}[labelindent=\widthof{{\code{mapattributes}}},leftmargin=*,noitemsep,nolistsep,align=right]
\item[\code{spacename}] The name of the space as a string or symbol.
\item[\code{key}] The key for the operation as a Ruby type
\item[\code{predicates}] A hash of predicates to check against.
\item[\code{mapattributes}] A hash specifying map attributes to modify and their respective key/values.
\end{description}

\noindent\textbf{Returns:}
True if predicate, False if not predicate.  Raises exception on error.

\paragraph{\code{map\_string\_append}}
\index{map\_string\_append!Ruby API}
\begin{ccode}
Client :: map_string_append(spacename, key, mapattributes)
\end{ccode}
\funcdesc Append the specified string to the value of the key-value pair for each map
attribute.
This operation requires a pre-existing object in order to complete successfully.
If no object exists, the operation will fail with \code{NOTFOUND}.



\noindent\textbf{Parameters:}
\begin{description}[labelindent=\widthof{{\code{mapattributes}}},leftmargin=*,noitemsep,nolistsep,align=right]
\item[\code{spacename}] The name of the space as a string or symbol.
\item[\code{key}] The key for the operation as a Ruby type
\item[\code{mapattributes}] A hash specifying map attributes to modify and their respective key/values.
\end{description}

\noindent\textbf{Returns:}
True.  Raises exception on error.

\paragraph{\code{cond\_map\_string\_append}}
\index{cond\_map\_string\_append!Ruby API}
\begin{ccode}
Client :: cond_map_string_append(spacename, key, predicates, mapattributes)
\end{ccode}
\funcdesc XXX


\noindent\textbf{Parameters:}
\begin{description}[labelindent=\widthof{{\code{mapattributes}}},leftmargin=*,noitemsep,nolistsep,align=right]
\item[\code{spacename}] The name of the space as a string or symbol.
\item[\code{key}] The key for the operation as a Ruby type
\item[\code{predicates}] A hash of predicates to check against.
\item[\code{mapattributes}] A hash specifying map attributes to modify and their respective key/values.
\end{description}

\noindent\textbf{Returns:}
True if predicate, False if not predicate.  Raises exception on error.

\paragraph{\code{search}}
\index{search!Ruby API}
\begin{ccode}
Client :: search(spacename, predicates)
\end{ccode}
\funcdesc Return all objects that match the specified \code{checks}.
This operation behaves as an iterator and may return multiple objects from the
single call.



\noindent\textbf{Parameters:}
\begin{description}[labelindent=\widthof{{\code{predicates}}},leftmargin=*,noitemsep,nolistsep,align=right]
\item[\code{spacename}] The name of the space as string or symbol.
\item[\code{predicates}] A hash of predicates to check against.
\end{description}

\noindent\textbf{Returns:}
Object if found, nil if not found.  Raises exception on error.

\paragraph{\code{search\_describe}}
\index{search\_describe!Ruby API}
\begin{ccode}
Client :: search_describe(spacename, predicates)
\end{ccode}
\funcdesc Return a human-readable string suitable for debugging search internals.  This
API is only really relevant for debugging the internals of \code{search}.


\noindent\textbf{Parameters:}
\begin{description}[labelindent=\widthof{{\code{predicates}}},leftmargin=*,noitemsep,nolistsep,align=right]
\item[\code{spacename}] The name of the space as a string or symbol.
\item[\code{predicates}] A hash of predicates to check against.
\end{description}

\noindent\textbf{Returns:}
Description of search.  Raises exception on error.

\paragraph{\code{sorted\_search}}
\index{sorted\_search!Ruby API}
\begin{ccode}
Client :: sorted_search(spacename, predicates, sortby, limit, maxmin)
\end{ccode}
\funcdesc XXX


\noindent\textbf{Parameters:}
\begin{description}[labelindent=\widthof{{\code{predicates}}},leftmargin=*,noitemsep,nolistsep,align=right]
\item[\code{spacename}] The name of the space as string or symbol.
\item[\code{predicates}] A hash of predicates to check against.
\item[\code{sortby}] The attribute to sort by.
\item[\code{limit}] The number of results to return.
\item[\code{maxmin}] Maximize (!= 0) or minimize (== 0).
\end{description}

\noindent\textbf{Returns:}
Object if found, nil if not found.  Raises exception on error.

\paragraph{\code{group\_del}}
\index{group\_del!Ruby API}
\begin{ccode}
Client :: group_del(spacename, predicates)
\end{ccode}
\funcdesc Delete all objects that match the specified \code{checks}.

This operation will only affect objects that match the provided \code{checks}.
Objects that do not match \code{checks} will be unaffected by the group call.
Each object that matches \code{checks} will be atomically updated with the check
on the object.  HyperDex guarantees that no object will be altered if the
\code{checks} do not pass at the time of the write.  Objects that are updated
concurrently with the group call may or may not be updated; however, regardless
of any other concurrent operations, the preceding guarantee will always hold.



\noindent\textbf{Parameters:}
\begin{description}[labelindent=\widthof{{\code{predicates}}},leftmargin=*,noitemsep,nolistsep,align=right]
\item[\code{spacename}] The name of the space as a string or symbol.
\item[\code{predicates}] A hash of predicates to check against.
\end{description}

\noindent\textbf{Returns:}
True if predicate, False if not predicate.  Raises exception on error.

\paragraph{\code{count}}
\index{count!Ruby API}
\begin{ccode}
Client :: count(spacename, predicates)
\end{ccode}
\funcdesc XXX


\noindent\textbf{Parameters:}
\begin{description}[labelindent=\widthof{{\code{predicates}}},leftmargin=*,noitemsep,nolistsep,align=right]
\item[\code{spacename}] The name of the space as a string or symbol.
\item[\code{predicates}] A hash of predicates to check against.
\end{description}

\noindent\textbf{Returns:}
Number of objects found.  Raises exception on error.


\subsection{Working with Signals}

Your application must mask all signals prior to making any calls into the
Ruby bindings.  The Ruby bindings will unmask the signals during blocking
operations and raise a \code{HyperDexClientException} with status
\code{'HYPERDEX\_CLIENT\_INTERRUPTED'} should any signals be received.

\subsection{Working with Threads}

The Ruby module is fully reentrant.  Instances of
\code{HyperDex::Client::Client} and their associated state may be accessed from
multiple threads, provided that the application employes its own synchronization
that provides mutual exclusion.

Put simply, a multi-threaded application should protect each \code{Client}
instance with a mutex or lock to ensure correct operation.
