\chapter{Tuning HyperDex}

Out of the box, HyperDex provides a stable system with decent performance.  This
chapter is dedicated to those who are looking to get even more performance out
of their system.  In this chapter, we'll explore system settings that can impact
HyperDex's performance.

\section{Filesystem}

By default, most filesystems are not tuned for optimal performance with
HyperDex.  In particular, many Linux filesystems update filesystem metadata
on each read or write operation, and will totally order all operations on disk,
adversely affecting performance.

\subsection{Access Time}

HyperDex stores its data in many files on disk, with each file containing a
small amount of the overall data.  Each read in HyperDex may access multiple
files to satisfy the read.  While reads are generally cheap, and may be served
from cache, most filesystems maintain information associated with each file
about the last time the file was accessed.  Many filesystems can write this
information to disk each time a file is accessed, converting relatively cheap
reads into more expensive write operations.

For that reason, we recommend that the volume containing the HyperDex daemons'
data directories be mounted with the ``noatime'' flag.  This prevents the
filesytem from recording last-accessed times on files with each read.  Although
the details are filesystem-dependent, the ``noatime'' flag is generally
configured by modifying \code{/etc/fstab} to contain the flag:

\begin{verbatim}
/dev/sda /hyperdex/data/dir ext4 defaults,noatime 0 2
\end{verbatim}

Consult your filesystem and distribution's documentation for the proper way to
disable access-time updates.

\subsection{Open Files}

Internally, HyperDex maintains multiple open file descriptors corresponding to
network sockets and data files.  Most Linux systems restrict the number of open
files to 1024 by default.  We recommend setting this value to 65536 or higher.
