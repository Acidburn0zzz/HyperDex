\chapter{Backup and Recovery}
\label{chap:backups}

Backups provide disaster recovery and protect against user error.  HyperDex's
replication provides some of this same protection, but there is no substitute
for proper backup hygiene.  A full blown backup protection mechanism protects
the cluster correlated failures that would otherwise violate HyperDex's
$f$-fault tolerance guarantees.

HyperDex provides one of the best backup experiences available in any NoSQL
store.  HyperDex's backups enable administrators to backup the on-disk state of
HyperDex servers in their native on-disk state.  As a result, backups are quick
and efficient, because there is no need to serialize and deserialize the data
during the backup process.  The backup/restore process is strongly consistent,
doesn't require shutting down servers, and enables incremental backup support.
Further, the process of initiating a backup is quite efficient; it completes
quickly, and does not consume CPU or I/O for extended periods of time.

\section{Consistent versus Inconsistent Backups}

In a distributed data store, the simple but flawed way to make backups is to
take a snapshot at each of the shards independently.  Needless to say, this is
precisely what a number of other systems do.  The problem with this approach is
that backups taken without coordination among the servers are not going to take
a consistent cut through the data.  A portion of the data in the backup might
come from a snapshot at time T0, while others portions come from T1, and yet
others from different points in time.  Depending on the application, these kinds
of uncoordinated backups can lead to inconsistencies in the database.  Restoring
from a backup might take the system to a state that actually never existed. And
that defeats the entire point of a backup.

In contrast, HyperDex backups are coordinated.  The network pauses for slightly
less than a second, the operation queues are drained, pending operations are
completed, and therefore the snapshots capture a consistent, clean view of the
data.  A restore will take the system back to a precise point in time. 

\section{Simple Backup Strategies}

The simplest strategy for taking a backup is to use the included
\code{backup-manager} command.  Of course, before you can take a backup of a
cluster, the cluster has to exist.  We can easily setup a simple cluster with by
executing following sequence of commands in separate terminals.

\begin{consolecode}
% hyperdex coordinator -f -l 127.0.0.1 -p 1982
% hyperdex daemon -f --listen=127.0.0.1 --listen-port=2012 \
                     --coordinator=127.0.0.1 --coordinator-port=1982 --data=/path/to/data
\end{consolecode}

This will start a single-node cluster on our local machine.  We can take a
backup of this cluster by issuing the following command:

\begin{consolecode}
% hyperdex backup-manager --backup-dir /path/to/backups
\end{consolecode}

Our backup of the entire cluster now resides under the \code{/path/to/backups}
directory.  Behind the scenes, this command will:

\begin{enumerate}
\item Put the cluster into read-only mode
\item Wait for ongoing writes to quiesce
\item Take a backup of the coordinator
\item Take a backup of each individual daemon
\item Put the cluster back into read-write mode
\item Copy the coordinator state and each of the daemon's state to
    \code{/path/to/backups/YYY-mm-ddTHH:MM:SS} for the current date and time,
    exploiting data available in previous backups if possible.  This relies on
    having ssh access to the listed hosts, and using rsync to copy the data.
\end{enumerate}

For example, after the above backup, the directory hierarchy will look like
this:

\begin{consolecode}
% find /path/to/backups
/path/to/backups
/path/to/backups/LATEST
/path/to/backups/2015-01-05T18:43:08
/path/to/backups/2015-01-05T18:43:08/coordinator.bin
/path/to/backups/2015-01-05T18:43:08/13036267341651542609
/path/to/backups/2015-01-05T18:43:08/13036267341651542609/000004.log
/path/to/backups/2015-01-05T18:43:08/13036267341651542609/000006.log
/path/to/backups/2015-01-05T18:43:08/13036267341651542609/MANIFEST-000002
/path/to/backups/2015-01-05T18:43:08/13036267341651542609/000005.sst
/path/to/backups/2015-01-05T18:43:08/13036267341651542609/CURRENT
/path/to/backups/2015-01-05T18:43:08/13036267341651542609/LOG
\end{consolecode}

We can see our recent backup is in directory \code{2015-01-05T18:43:08}.  Within
this directory is a backup of the coordinator state, \code{coordinator.bin}, and
one directory for the server's state.  Because we only have one server, we have
one directory.  With more servers, we would see more directories, each
identified by the server's unique token.  We can get a list of the servers in
our cluster with this command:

\begin{consolecode}
% hyperdex show-config | grep server
\end{consolecode}

We can see that indeed, the only server in our cluster is
\code{13036267341651542609}.  The files stored within this directory are a copy
of the on-disk data at our server taken atomically during the backup process.

Suppose the unthinkable happens (a datacenter outage, cooling failure, flood,
act of God, or more likely, some human error that causes data loss) we need to
restore from backup on a different set of machines.  Restoring the cluster is as
simple as copying the backup directory
\code{/path/to/backups/2015-01-05T18:43:08/13036267341651542609} to a different
daemon server, and then launching a new coordinator from the backup:

\begin{consolecode}
% hyperdex coordinator --restore coordinator.bin --foreground -l 127.0.0.1 -p 1983
\end{consolecode}

We can then bring the remainder of the cluster online by using the copied data
directory on an entirely different server from the one that initially existed in
our cluster.

\begin{consolecode}
% hyperdex daemon -f --listen=127.0.0.1 --listen-port=2013 \
                     --coordinator=127.0.0.1 --coordinator-port=1983 --data=/path/to/copy
\end{consolecode}

All freshly-restored clusters come online in read-only mode.  Once all daemons
have come online, you can make the cluster fully operational by executing:

\begin{consolecode}
% hyperdex set-read-write -h 127.0.0.1 -p 1983
\end{consolecode}

At this point, we have two fully functional HyperDex clusters running.  We have
our original cluster with the coordinator on port 1982, and the daemon on 2012;
and, we have the restored copy of the cluster running on on ports 1983 and 2013.

\section{Backup Efficiency}

HyperDex's backups are extremely efficient, because they leverage the structure
of the on-disk format to take snapshots quickly, which may then be copied in the
background.

HyperDex uses HyperLevelDB as its storage backend, which, in turn, constructs an
LSM-tree on disk.  The majority of data stored within HyperLevelDB is stored
within immutable \code{.sst} files.  Once written, these files are never
overwritten.  The backup feature within HyperDex can extract a snapshot of these
files via an instantaneous hard link in the filesystem. This state, once
snapshotted so efficiently, can then be transferred elsewhere at will. 

At the cluster level, backups pause writes for a short period of time while the
HyperDex daemons take their individual snapshots.  Because individual snapshots
are efficient, the pause generally takes less than a second.

Finally, the backup tool ensures that the cluster is restored to full operation
before transferring any data over the network.  This transfer comprises the bulk
of the backup operation, and happens only after the cluster resumes normal
operation.

\section{Advanced Backup Techniques}

Backup solutions often require customization; some setups use network attached
storage and back up to a centralized location, others will demand custom control
over where and how the snapshots are transferred.  Let's walk through a more
detailed example where the snapshots are manually managed.

Every custom backup solution will start by initiating a snapshot operation
within the HyperDex cluster.  The command to take a snapshot is called
\code{backup}.  For example, this command will create a snapshot named
\code{my-backup}:

\begin{consolecode}
% hyperdex backup my-backup
13036267341651542609 127.0.0.1 /path/to/data/backup-my-backup
\end{consolecode}

When this command succeeds, it will exit zero, and output a list of the servers
that hold individual snapshots.  For instance, we can see that our server
identified by \code{13036267341651542609} on 127.0.0.1 backed up its state to
\code{13036267341651542609}.

We can then backup this state to wherever we wish to store the backup.  In the
backup-manager command, we transfer the state to the specified directory using
rsync like this:

\begin{consolecode}
% rsync -a --delete -- 127.0.0.1:/path/to/data/backup-my-backup/ 13036267341651542609/
\end{consolecode}

This command will create a directory named ``13036267341651542609'' and transfer
all of the server's state to this directory.  This directory is suitable for
passing to the \code{--data} parameter of HyperDex's daemon.

In addition to the per-daemon state backups, the backup command above also
creates a file in the curren directory called \code{my-backup.coordinator.bin}.
This file contains the coordinator's state, suitable for using to bootstrap a
new coordinator cluster as shown above.

The backup-manager command shown above builds upon this low-level backup
command, and neatly organizes the backup of each server and the coordinator into
the backup directory.  Of course, other possible schemes include:

\begin{itemize}
\item Using filesystem-level snapshots to save the backup directories.  Daemons
may then be launched from the snapshots in the future.  While this protects
against some software faults, it doesn't provide as thorough protection against
disaster recovery as moving the backups offsite.

\item  Having the daemons directly rsync their state to an off-site storage
location.  This avoids any resource constraints that may limit the effectiveness
of the backup-manager, and avoids the possibility for hardware failure to
corrupt filesystem-level snapshots.

\item Restore multiple instances of a cluster simultaneously to experiment with
your production data.  For example, before deploying a new application to
production, backup the cluster, and restore the backup in parallel to the
production cluster.  Then, deploy the new application against the parallel
cluster before deploying to production.
\end{itemize}

\section{Incremental Backups with Rsync}

The backup-manager command demonstrated above supports {\em incremental
backups}, wherein successive calls will only transfer state that has been
modified since the last backup.  Each backup appears as a full backup of the
cluster and shares storage resources with previous backups.  Consequently, the
backup manager will store several complete backups of the cluster without
requiring the storage space of several complete backups.

The implementation leverages the \code{--link-dest} option to rsync to avoid
making redundant copies of the data.  When provided this option, rsync will look
for files in the specified directory.  If the file in the source directory is
also present, unchanged, in the link-dest directory, then it will be hard-linked
into place and not copied over the network.

Assuming an existing backup exists in the directory
\code{13036267341651542609-1},
we can create an incremental backup in \code{13036267341651542609-2} by issuing:

\begin{consolecode}
% rsync -a --delete --link-dest=13036267341651542609-1 -- \
    127.0.0.1:/path/to/data/backup-my-backup/ \
    13036267341651542609-2
\end{consolecode}

This will use rsync to de-duplicate the data and avoid redundant copies.  Other
possibilities include storing the data into a storage service like S3, with a
higher level application orchestrating the de-duplication logic.
