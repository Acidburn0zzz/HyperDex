\chapter{Transactions}

% .. _transactions:
% 
% Transactions
% ============
% 
% HyperDex Warp is a superset of HyperDex that provides transactions.  For this
% chapter, you'll need to install the hyperdex-warp demo package available at
% http://hyperdex.org/
% 
% By the end of this chapter you'll be familiar with HyperDex Warp's transactions.
% 
% Setup
% -----
% 
% As in the previous chapter, the first step is to deploy the cluster and connect
% a client.   First we launch and initialize the coordinator:
% 
% .. sourcecode:: console
% 
%    $ hyperdex coordinator -f -l 127.0.0.1 -p 1982
% 
% Next, let's launch a few daemon processes to store data.  Execute the following
% commands (note that each instance binds to a different port and has a different ``/path/to/data``):
% 
% .. sourcecode:: console
% 
%    $ hyperdex daemon -f --listen=127.0.0.1 --listen-port=2012 \
%                         --coordinator=127.0.0.1 --coordinator-port=1982 --data=/path/to/data1
%    $ hyperdex daemon -f --listen=127.0.0.1 --listen-port=2013 \
%                         --coordinator=127.0.0.1 --coordinator-port=1982 --data=/path/to/data2
%    $ hyperdex daemon -f --listen=127.0.0.1 --listen-port=2014 \
%                         --coordinator=127.0.0.1 --coordinator-port=1982 --data=/path/to/data3
% 
% 
% This brings up three different daemons ready to serve in the HyperDex cluster.
% Finally, we create a space which makes use of all three systems in the cluster.
% In this example, let's create a space that may be suitable for storing virtual
% currency in a banking application:
% 
% .. sourcecode:: console
% 
%    >>> import hyperclient
%    >>> c = hyperclient.Client('127.0.0.1', 1982)
%    >>> c.add_space('''
%    ... space accounts
%    ... key id
%    ... attribute
%    ...    int balance
%    ... ''')
% 
% Let's create some accounts with some starting balances:
% 
% .. sourcecode:: pycon
% 
%    >>> c.put('accounts', 'joe', {'balance': 100})
%    True
%    >>> c.put('accounts', 'brian', {'balance': 25})
%    True
% 
% Using Transactions
% ------------------
% 
% The prototypical example of where transactions come in handy is a standard bank
% debit/credit scenario.  If Joe wanted to transfer currency to Brian, it would be
% bad for Joe and Brian if the money were to leave Joe's account and not find its
% way into Brian's.  It would be bad for the bank if money were to be created out
% of thin air by a deposit in Brian's account without a corresponding withdrawal
% from Joe's -- manipulating markets in such a fashion is reserved for those who
% are too big to fail.
% 
% In the prototypical, "Hello World," example involving transactions, we transfer
% $10 from Joe to Brian:
% 
% .. sourcecode:: pycon
% 
%    >>> x = c.begin_transaction()
%    >>> brian = x.get('accounts', 'brian')['balance']
%    >>> joe = x.get('accounts', 'joe')['balance']
%    >>> x.put('accounts', 'brian', {'balance': brian + 10})
%    True
%    >>> x.put('accounts', 'joe', {'balance': joe - 10})
%    True
%    >>> x.commit()
%    True
%    >>> print "Brian's balance =", c.get('accounts', 'brian')['balance']
%    Brian's balance = 35
%    >>> print "Joe's balance =", c.get('accounts', 'joe')['balance']
%    Joe's balance = 90
% 
% This transaction is relatively straight-forward.  There are three critical
% properties that will distinguish this transaction.  One, either both of these
% operations execute, or neither do.  It is impossible for half of a transaction
% to be lost.
% 
% Two, there is no eventual consistency.  All transaction results are immediately
% visible to all future events.  There is no need to reconcile or maintain
% multiple versions.  There is no inconsistency.
% 
% Finally, HyperDex Warp is fault tolerant.  A user-configurable fault tolerance
% threshold, ``f`` allows HyperDex Warp to remain available in the presence of
% concurrent failures.
% 
% For a real bank application, we'd likely want to charge Joe an (excessive)
% overdraft fee if he didn't have the money to cover the transfer.  With
% transactions, implementing this case is trivial:
% 
% .. sourcecode:: pycon
% 
%    >>> x = c.begin_transaction()
%    >>> brian = x.get('accounts', 'brian')['balance']
%    >>> joe = x.get('accounts', 'joe')['balance']
%    >>> x.put('accounts', 'brian', {'balance': brian + 10})
%    True
%    >>> if joe < 10:
%    ...     # assess an overdraft fee on Joe
%    ...     joe -= 35
%    ...
%    >>> x.put('accounts', 'joe', {'balance': joe - 10})
%    True
%    >>> x.commit()
%    True
%    >>> print "Brian's balance =", c.get('accounts', 'brian')['balance']
%    Brian's balance = 45
%    >>> print "Joe's balance =", c.get('accounts', 'joe')['balance']
%    Joe's balance = 80
% 
% Here, we've successfully transferred money from Joe to Brian and cover the case
% where the bank wishes to charge Joe a fee for not having enough money in the
% first place.
% 
% The astute reader will notice that the example above is very different from
% doing the following:
% 
% .. sourcecode:: pycon
% 
%    >>> # an example of broken code
%    >>> c.atomic_add('accounts', 'brian', {'balance': 10})
%    True
%    >>> c.atomic_sub('accounts', 'joe', {'balance': 10})
%    True
%    >>> print "Brian's balance =", c.get('accounts', 'brian')['balance']
%    Brian's balance = 55
%    >>> print "Joe's balance =", c.get('accounts', 'joe')['balance']
%    Joe's balance = 70
% 
% This example is broken in multiple ways.  First, even though each individual
% operation is atomic, the two operations are not indivisible, and will be
% interspersed with all other operations on the accounts.  Second, the client is a
% single point of failure for this transaction.  Imagine if the server executing
% these operations failed between the two atomic operations.  No matter what the
% order of operations, someone would lose money.
% 
% Transactions are especially useful when multiple competing processes are
% executing transactions concurrently.  HyperDex Warp guarantees *one-copy
% serializablility*.  Any number of transactions may execute simultaneously.
% The final state of the database will always be identical to executing the
% committed transactions sequentially without concurrency.
% 
% Let's illustrate HyperDex's behavior with concurrent transactions.
% In the same terminal, let's begin a new transaction to pay Joe his yearly
% 5% interest payment:
% 
% .. sourcecode:: pycon
% 
%    >>> x = c.begin_transaction()
%    >>> balance = x.get('accounts', 'joe')['balance']
%    >>> balance = int(balance * 1.05)
%    >>> x.put('accounts', 'joe', {'balance': balance})
%    True
% 
% At this point, transaction ``x`` has not committed.  Any modifications performed
% within the context of a transaction are visible only within the transaction.  No
% other clients or transactions may observe the state.  Verify this for yourself
% with:
% 
% .. sourcecode:: pycon
% 
%    >>> print "Joe's balance =", c.get('accounts', 'joe')['balance']
%    Joe's balance = 70
%    >>> print "Joe's balance =", x.get('accounts', 'joe')['balance']
%    Joe's balance = 73
% 
% Joe's balance is still the same as before.  Now, open another window and start a
% second, concurrent transaction where Joe deposits cash at a branch location:
% 
% .. sourcecode:: pycon
% 
%    >>> import hyperclient
%    >>> c2 = hyperclient.Client('127.0.0.1', 1982)
%    >>> y = c2.begin_transaction()
%    >>> balance = y.get('accounts', 'joe')['balance']
%    >>> balance += 386
%    >>> y.put('accounts', 'joe', {'balance': balance})
%    True
% 
% At this point, both ``x`` and ``y`` are ready to commit, but neither has
% actually altered the account balance.  In fact, there is no way for ``x`` and
% ``y`` to commit that doesn't violate consistency.  If ``x`` commits before
% ``y``, the ending balance will be $466, shorting Joe of the $4 in interest he
% is due.  If, however, ``y`` commits before ``x``, then the ending balance of $84
% will cause Joe's cash deposit of $386 to be lost into the ether.  In this
% situation, one transaction must abort.  If ``y`` commits first, then ``x`` will
% abort:
% 
% In the second terminal, commit ``y``:
% 
% .. sourcecode:: pycon
% 
%    >>> y.commit()
%    True
%    >>> print "Joe's balance =", c2.get('accounts', 'joe')['balance']
%    Joe's balance = 456
% 
% Trying to commit ``x`` in the first terminal will result in the transaction
% being aborted by HyperDex Warp:
% 
% .. sourcecode:: pycon
% 
%    >>> x.commit()
%    Traceback (most recent call last):
%    HyperClientException: HyperClient(HYPERCLIENT_ABORTED, Transaction was aborted)
% 
% As expected, transaction ``x`` will not alter Joe's balance:
% 
% .. sourcecode:: pycon
% 
%    >>> print "Joe's balance =", c2.get('accounts', 'joe')['balance']
%    Joe's balance = 456
% 
% A transaction will only abort if it was executed simultaneously with another
% transation that operates on the same data.  The typical process for handling
% aborted transactions is to repeatedly try the transaction until it succeeds:
% 
% .. sourcecode:: pycon
% 
%    >>> committed = False
%    >>> while not committed:
%    ...     try:
%    ...         x = c.begin_transaction()
%    ...         balance = x.get('accounts', 'joe')['balance']
%    ...         balance = int(balance * 1.05)
%    ...         x.put('accounts', 'joe', {'balance': balance})
%    ...         committed = x.commit()
%    ...     except hyperclient.HyperClientException as e:
%    ...         if e.status != hyperclient.HYPERCLIENT_ABORTED:
%    ...             raise e
%    ...
%    True
% 
% Rich API
% --------
% 
% HyperDex Warp Transactions expose the full set of atomic and asynchronous
% operations from the HyperDex API.  All operations performed under a transaction
% are fully transactional.  The following example shows asynchronous atomic math
% operations combined with an asynchronous commit:
% 
% .. sourcecode:: pycon
% 
%    >>> t = c.begin_transaction()
%    >>> d1 = t.async_atomic_add('accounts', 'brian', {'balance': 10})
%    >>> d2 = t.async_atomic_sub('accounts', 'joe', {'balance': 10})
%    >>> d1.wait()
%    True
%    >>> d2.wait()
%    True
%    >>> d3 = t.async_commit()
%    >>> d3.wait()
%    True
% 
% All transactions operate on the most recent copy of the data.  When a
% transaction commits, the transaction is cosistently applied.  HyperDex Warp
% really means it when it says a transaction has committed.  There are no
% conflicting operations to reconcile later (after the system commits), and there
% is no eventual consistency.  All effects are immediate and permanent.
% 
% Summary
% -------
% 
% HyperDex Warp provides decentralized, distributed ACID transactions.  Committed
% transactions are one-copy serializable, making it extremely easy to reason about
% the concurrent operations.  The rich API coupled with transactional guarantees
% provide a rare combination of ease-of-use, correctness, and performance that
% other NoSQL systems cannot provide.
% 
% Get your copy of `HyperDex Warp`_ today.
% 
% .. _HyperDex Warp: http://hyperdex.org/Warp
% 
% .. todo::
% 
%    .. sourcecode:: pycon
% 
%       >>> c.rm_space('accounts')
